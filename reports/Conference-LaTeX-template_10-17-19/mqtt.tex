\section{MQTT}
Das \textbf{Message Queuing Telemetry Transport} oder kurz \textbf{MQTT} ist ein Protokoll was für die Kommunikation zwischen Maschinen zum Einsatz kommt. Somit eignet sich dieses Protokoll auch im wesentlichen für seinen Einsatz im Anwendungsbereich der Automatisierung und ganz besonders im Bereich des Internet der Dinge (\textbf{IoT} Internet of Things), da aufgrund seines Aufbaus Geräte mit wenig Akkukapazität so gut wie keine eigene Rechenleistung erbringen müssen um Nachrichten zu empfangen oder zu Senden ist hier ein deutlicher Vorteil gerade bei Akku betriebenen Systemen zu beobachten.\\
Erreicht wird dies durch den generellen Aufbau des Protokolls mittels subscriber und einem zentralen Broker, wobei der Broker der verwaltende und Daten haltende Part ist und die subscriber sowohl Nachrichten empfangen und senden können. Ein weiterer ganz wesentlicher vorteil des MQTT protokolls liegt in seiner struktur als publishing and Subcribe verfahren, durch das senden der nachrichten nicht als direktverbindung der clients unternander, sondern mit dem umweg über den Broker sind die geräte in der lage nachrichten sicher zu empfangen auch wenn sie gerade nicht auf nachrichten empfang stehen.\cite{b1}
