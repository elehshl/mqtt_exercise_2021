\documentclass[conference]{IEEEtran}
\IEEEoverridecommandlockouts
% The preceding line is only needed to identify funding in the first footnote. If that is unneeded, please comment it out.
\usepackage{cite}
\usepackage{amsmath,amssymb,amsfonts}
\usepackage{algorithmic}
\usepackage{graphicx}
\usepackage{textcomp}
\usepackage{xcolor}
\def\BibTeX{{\rm B\kern-.05em{\sc i\kern-.025em b}\kern-.08em
    T\kern-.1667em\lower.7ex\hbox{E}\kern-.125emX}}
\begin{document}

\title{Conference Paper Title*\\
{\footnotesize \textsuperscript{*}Note: Sub-titles are not captured in Xplore and
should not be used}
\thanks{Identify applicable funding agency here. If none, delete this.}
}

\author{\IEEEauthorblockN{1\textsuperscript{st} Justin Frommberger}
\IEEEauthorblockA{\textit{Interaktionstechnik und Design} \\
\textit{Hochschule Hamm-Lippstadt}\\
City, Country \\
email address or ORCID}
\and
\IEEEauthorblockN{2\textsuperscript{nd} Jonas Gerken }
\IEEEauthorblockA{\textit{Interaktionstechnik und Design} \\
\textit{Hochschule Hamm-Lippstadt}\\
City, Country \\
email address or ORCID}
\and
\IEEEauthorblockN{3\textsuperscript{rd} Benedikt Lipinski}
\IEEEauthorblockA{\textit{Interaktionstechnik und Design} \\
\textit{Hochschule Hamm-Lippstadt)}\\
Soest, Deutschland \\
benedikt.lipinski@stud.hshl.de}
\and
\IEEEauthorblockN{4\textsuperscript{th} Phillip Wagner}
\IEEEauthorblockA{\textit{Interaktionstechnik und Desgin} \\
\textit{Hochschule Hamm-Lippstadt}\\
City, Country \\
email address or ORCID}
}

\maketitle

\begin{abstract}
This document is a model and instructions for \LaTeX.
This and the IEEEtran.cls file define the components of your paper [title, text, heads, etc.]. *CRITICAL: Do Not Use Symbols, Special Characters, Footnotes, 
or Math in Paper Title or Abstract.
\end{abstract}

\begin{IEEEkeywords}
component, formatting, style, styling, insert
\end{IEEEkeywords}

\newpage
\section{Parts of Integration}
\section{Server}
In diesem Kontext spielt der Server eine sehr wichtige rolle in der Kommunikation zwischen den ausführenden parts des Projektes. Durch den Server und seine strukturen wird letztendlich erst eine plattform geschaffen, die Allen fahrezugen und den Kunden (Usern) eine Möglichkeit bietet eine verbindung  unter einander zu schaffen. Konkret waren die aufgaben des Servers : 
\begin{table}[h]
\begin{tabular}{lcr}
Anmeldung von clients\\
Das nächstgelegene Fahrzeug finden\\
Jedem client eine eindeutige ID zuordnen \\
Übermittelt fahrzeugdaten an den Kunden\\
Interne verarbeitung einer fahrzeug bestellung\\

\end{tabular}
\end{table}

\begin{thebibliography}{00}
\bibitem{b1} 
\end{thebibliography}
\vspace{12pt}
\end{document}
