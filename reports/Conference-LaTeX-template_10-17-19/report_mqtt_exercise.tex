\documentclass[conference]{IEEEtran}
\IEEEoverridecommandlockouts
% The preceding line is only needed to identify funding in the first footnote. If that is unneeded, please comment it out.
\usepackage{cite}
\usepackage{amsmath,amssymb,amsfonts}
\usepackage{algorithmic}
\usepackage{graphicx}
\usepackage{textcomp}
\usepackage{xcolor}
\usepackage{listings}
\def\BibTeX{{\rm B\kern-.05em{\sc i\kern-.025em b}\kern-.08em
    T\kern-.1667em\lower.7ex\hbox{E}\kern-.125emX}}

\definecolor{keywords}{RGB}{255,0,90}
\definecolor{comments}{RGB}{0,0,113}
\definecolor{red}{RGB}{160,0,0}
\definecolor{green}{RGB}{0,150,0}
\begin{document}

\lstset{language=Python, 
        basicstyle=\ttfamily\small, 
        keywordstyle=\color{keywords},
        commentstyle=\color{comments},
                stringstyle=\color{red},
        showstringspaces=false,
        identifierstyle=\color{green},
        keywords=[2]{pow},
        keywordstyle=[2]{\color{orange}},
}
\lstset{frame=lines}
%\lstset{caption={Insert code directly in your document}}
\lstset{label={lst:code_direct}}
\lstset{basicstyle=\footnotesize}








\title{Conference Paper Title*\\
{\footnotesize \textsuperscript{*}Note: Sub-titles are not captured in Xplore and
should not be used}
\thanks{Identify applicable funding agency here. If none, delete this.}
}

\author{\IEEEauthorblockN{1\textsuperscript{st} Justin Frommberger}
\IEEEauthorblockA{\textit{Interaktionstechnik und Design} \\
\textit{Hochschule Hamm-Lippstadt}\\
City, Country \\
email address or ORCID}
\and
\IEEEauthorblockN{2\textsuperscript{nd} Jonas Gerken }
\IEEEauthorblockA{\textit{Interaktionstechnik und Design} \\
\textit{Hochschule Hamm-Lippstadt}\\
City, Country \\
email address or ORCID}
\and
\IEEEauthorblockN{3\textsuperscript{rd} Benedikt Lipinski}
\IEEEauthorblockA{\textit{Interaktionstechnik und Design} \\
\textit{Hochschule Hamm-Lippstadt)}\\
Soest, Deutschland \\
benedikt.lipinski@stud.hshl.de}
\and
\IEEEauthorblockN{4\textsuperscript{th} Phillip Wagner}
\IEEEauthorblockA{\textit{Interaktionstechnik und Desgin} \\
\textit{Hochschule Hamm-Lippstadt}\\
City, Country \\
email address or ORCID}
}

\maketitle

\begin{abstract}
This document is a model and instructions for \LaTeX.
This and the IEEEtran.cls file define the components of your paper [title, text, heads, etc.]. *CRITICAL: Do Not Use Symbols, Special Characters, Footnotes, 
or Math in Paper Title or Abstract.
\end{abstract}

\begin{IEEEkeywords}
component, formatting, style, styling, insert
\end{IEEEkeywords}

\newpage
\section{Parts of Integration}
\section{Server}
In diesem Kontext spielt der Server eine sehr wichtige rolle in der Kommunikation zwischen den ausführenden parts des Projektes. Durch den Server und seine strukturen wird letztendlich erst eine plattform geschaffen, die Allen fahrzeuge und den Kunden (Usern) eine Möglichkeit bietet eine verbindung  unter einander zu schaffen und weitere Aufgaben zu erledigen. Konkret waren die aufgaben des Servers: 
\begin{table}[h]
\begin{tabular}{lcr}
Anmeldung von clients\\
Das nächstgelegene Fahrzeug finden\\
Jedem client eine eindeutige ID zuordnen \\
Übermittelt fahrzeugdaten an den Kunden\\
Interne verarbeitung einer fahrzeug bestellung\\
\end{tabular}
\end{table}
\subsection{Anmedlung von Clients}
Aufgabe des Servers ist es eine Anmeldung von clients zu ermöglichen um  einerseits nur aus dem pool der aktuell aktiven,freie fahrzeuge  auszuwählen und andererseits einer unbekannten menge an Fahrzeugen und kunden die Möglichkeit zu bieten am angebot teilzuhaben. Zu den Clients gehören sowohl die Kunden (User), wie auch alle Fahrzeug typen , damit sind alle Servicefahrzeuge mit den unterkategorien: Police, Firefighter, ambulance gemeint und zuletzt auch Fahrzeuge der Kategorie Taxi.\\ 
Um auf eine Unbekannte Anzahl an sich zu regestrierenden clients reagieren zu können, muss der initiative schritt durch den Client erfolgen. Zur Regestrierung sendet der Client eine nachricht mittels mqtt Protokoll über die adressen des Users oder der Fahrzeug typen mit dem ersten teil \textsf{"adresse = hshl/mqtt\_exercise/"} und der endung des jeweiligen fahrzeug typen, also: \textsf{"user,taxi,police,firefighter,ambulance"} an den Server. Dieser verarbeitet die nachricht bei erhalt in der Funktion  
\begin{lstlisting}
def receive()
	....
	messageprocessing(temp)
	....
\end{lstlisting}
 wobei die aufgabe der Funktion  \textit{receive()} eher die generelle verarbeitung der mqtt nachricht darstellt und nicht die zuordnung der nachricht zu einem bestimmten anwendungszweck. 
Dieser wird anschließend durch den aufruf der Funktion \textit{messageprocessing} und einer teilung des übergebenen Arrays in die informationen zu inhalt und Adresse der nachricht erreicht.
um ein auslesen der nachricht überhaupt zu ermöglichen, ist es notwending die Empfangenen nachricht erst in das Json format zu codieren um anschließen die vorteile einer verarbeitung mit Json datasets zu nutzen. dies wird mit der textzeitle \begin{lstlisting}
def messageprocessing(msg)
	json.loads(str(msg[1]))
\end{lstlisting}
 erreicht.\\
 Zur besseren Identifizierung und klassifizierung als Regestierungsnachricht, wird in dieser die ID des clients durch das wort "register" ersetzt. Dies dient wie schon erwähnt einerseits der besseren einordnung, anderer seits half das menschen lesbarhalten von nachrichten bei der entwicklung ungemein. Einen negativen einfluss auf den erfolgreichen ablauf der regestrierung des clients hat dies nicht, da eine ID erst mit antwort des servers an den client vergeben wird. Die nachricht, die ein Client zur Regestrierung/Anmeldung senden muss sieht wie folgt aus:
\begin{lstlisting}
    data={
    "id": "register",
    "name": name,
    "coordinates":coor
    }	
\end{lstlisting}
Zudem wird in der Internen verarbeitung ein neuer kanal für die weitere kommunikation mit dem Client geschaffen, so dass eine direkte kommunikation mit diesem möglich ist ohne das andere clients hierdurch beeinträchtigt werden. Die adresse des neuen kanal wird unter zuhilfe nahme der druch den server in den funktionen \textit{regestrationUser(data)} und \textit{registrationCar(data, type)} vergebenen ID geöffnen und mittels einer nachricht auf dem kanal \textsf{"hshl/mqtt\_exercise/user/back"} an diesen zurück gesendet. \\
Anhand des Quellscodes für das regestrieren des users wird gezeigt, wie die vergabe einer Neuen ID und das einspeichern des clients in den server funktioniert, dies ist ganz ähnlich für das vorgenen bei der regestrierung von fahrzeugen, mit dem einzigen unterschied, das in diesem fall mittels einer separierung durch den übergabe wert "type" die einzelnen fahrzeuge unterschieden werden können. Desweiteren wird bei der registrierung der fahrzeuge noch der status "free" vergeben.\\
zuerst wird durch den aufruf der Funktion \textit{findid(user)} die kleinste noch freie ID aus der liste der angemeldeten fahrzeuge gesucht, indem die höchste vergebene ID gesucht wird und um einen zähler höher zurück gegeben wird. Anschliessend wird in der methode \textit{registrationUser(data)} durch ein weiteres durchlaufen der liste geprüft ob bereits ein user mit dem selben namen vorhanden ist und bei negativem ergebnis in  die liste aller user eingetragen.\\ der user bekommt abschliessend auf dem rückkanal eine nachricht mit seiner eindeutigen ID. 
\subsection{Bestellen eines Fahrzeugs}

\begin{thebibliography}{00}
\bibitem{b1} 
\end{thebibliography}
\vspace{12pt}
\end{document}
