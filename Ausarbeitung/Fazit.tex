\section{Fazit}
Das Ziel dieses Projekts bestand darin ein Programm zu entwickeln, welches dazu dient eine Person (hier auch User oder Nutzer) mithilfe einfachster Schritte einen speziellen Service in Anspruch nehmen zu lassen, welche sich aus dem vorherigen Kapitel unter dem Begriff Servicefahrzeuge entnehme lassen. Hierbei wurden diverse Anforderungen schon vorher festgelegt und definierten den Projektrahmen. So wurde gefordert, dass jeder Nutzer dieser Anwendung, mittels einer eigenen ID identifiziert werden kann und zudem seinen Namen und seine aktuellen Koordinaten an den Server schickt. Der Server wiederum diente der Verarbeitung und Speicherung der erhaltenen Nachrichten, um so eine Art 'Schwarzes Brett' für alle Services darzustellen. Die Servicefahrzeuge waren somit in der Lage sich die nötigen Informationen zu holen und mit dem Nutzer in Kontakt zu treten.\\
Im Bezug auf die Zukunft und die mögliche Anwendbarkeit einer solchen Anwendung lässt sich bisher nur spekulieren. Für einfache Services wie Taxen, welches ebenfalls eine Dienstleistung in dieser Anwendung ist, ist diese Art von Technik bereits durch ein namhaftes Unternehmen weltweit vertreten. Hinsichtlich der zur Verfügung Stellung von Polizei, Krankenwagen und Feuerwehr müssen jedoch noch viele Ressourcen in ein flächendeckendes und einheitliches System gesteckt und unter Berücksichtigung der Sicherheit beurteilt werden.\\
In diesem Fall war das Projekt jedoch ein Erfolg und könnte die Basis für ein eben solches System darlegen. Durch die gute Koordinierung und Rücksprache ließen sich Probleme und Anregungen schnell besprechen und in der Gruppe lösen.