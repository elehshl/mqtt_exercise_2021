\section{Einleitung}
Dadurch das Mqtt heutzutage vielseitig Einsatz findet,
wie für Sensordaten, klassische Nachrichten, 
Aktienkurse oder Kurzmitteilungen,
wie bei Facebooks Mobiler App. 


Dies weckt Interesse bei vielen Nutzern, 
sich mit dem Thema Mqtt vertraut zu machen.


MQTT ist ein Client-Server-Protokoll, 

Clients senden dem Server (Broker) 
nach Verbindungsaufbau Nachrichten mit einem Topic, 

welches die Nachrichten strukturiert, 

z.B. Haus,Wohnzimmer,Sofa.


So hat sich unsere Gruppe die Aufgabe gestellt, 
das Mqtt Projekt mqttexercise2021 zu Programmieren und so Kenntnisse zum Thema Mqtt zu sammeln.


In dem Projekt haben wir die Aufgabe erhalten, 
einen Server zu Programmieren der Nachrichten von anderen Clients (User, Services, Taxi) empfangen und diesen auch etwas zusenden kann.


\subsection{Aufgaben}

Server: Registriert alle Clients, verteilt IDS, übermittelt Nachrichten, 
erhält Nachrichten, leitet/organisiert den Ablaufplan.


User: Anmelden beim Server, order Fahrzeug, sendet Koordinaten, Fahrzeug kommt, wird zum Zielort gebracht


Taxi: Anmeldung Server, fährt zu den Koordinaten, 

übermittelt Koordinaten, übermittelt Nachrichten, erhält Nachrichten.


Services: Anmelden beim Server, sendet Koordinaten, 

warten auf Nachricht, 

bekommt Koordinaten von User, zum User fahren, 

Fahrzeug ist  bussy, User zum Ziel fahren, Fahrzeug wieder free

\subsection{Ziel}

Das Ziel des Projektes ist es, ein Programm zu schreiben, welches ein Mqtt Ablauf zeigt und eine funktionierende Kommunikation zwischen den Clients enthält.

\subsection{Ablauf}
Zunächst wird in Kapitel 2 die Projektaufgabe beschrieben. In Kapitel 3 werden die Vorbereitungen und Recherche dargestellt und in Kapitel 4 der Aufbau des Projektes. Die Arbeit endet mit einem Fazit. 






