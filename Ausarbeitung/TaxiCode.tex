\section{TaxiCode}

Die Aufgabe des Taxis ist es über den MQQT Server mit dem Server und dem User zu kommunizieren.
 
Über die erhaltenen Messages fährt das Taxi zum User und bringt ihn zu seinem Zielort.

Der erste Schritt vom Taxi ist, sich beim Server mit seinem Namen zu registrieren. 

Folgend weißt der Server dann dem Taxi seine ID zu.

\begin{lstlisting} 

data = {
	"id": "register", 
	"name": name,
	"coordinates": coor
    }
    
\end{lstlisting}



Daraufhin erhält das Taxi eine Message vom Server, dass ein User ein Taxi bestellt hat. 
Nun liest das Taxi die Koordinaten vom User aus. 

\begin{lstlisting} 
        data = {
        "id":id,
        "msg": "Arrival",
        "coordinates": js['coordinates']
        }
\end{lstlisting}



Mit den erhaltenden Koordinaten, weiß das Taxi an welcher Stelle sich der User befindet und fährt zu seiner aktuellen Position.
Am Zielort angekommen, wartet das Taxi auf die neue Message vom User, mit seinem neuen Zielort.


\begin{lstlisting} 
        data ={
        "id":id,
        "msg": "Arrival at destination",
        "coordinates": js['destination']
        }
\end{lstlisting}



Zum Schluss erhält das Taxi eine Anfrage vom Server, mit seiner neuen Position und sendet diese dann zum Server.

\begin{lstlisting} 
data={
        "id":id,
        "name":name,
        "coordinates":coor
        }
        send(json.dumps(data),
        "hshl/mqtt_exercise/set_position")

\end{lstlisting}
