\documentclass[conference]{IEEEtran}
\IEEEoverridecommandlockouts
% The preceding line is only needed to identify funding in the first footnote. If that is unneeded, please comment it out.
\usepackage{listings}
\lstset{
breaklines=true
}
\usepackage{cite}
\usepackage{amsmath,amssymb,amsfonts}
\usepackage{algorithmic}
\usepackage{graphicx}
\usepackage{textcomp}
\usepackage{xcolor}
\def\BibTeX{{\rm B\kern-.05em{\sc i\kern-.025em b}\kern-.08em
    T\kern-.1667em\lower.7ex\hbox{E}\kern-.125emX}}
\begin{document}
\lstset{language=Python}

\section{User Code}\label{usercode}
Der Code des Users wurde hinsichtlich der Anforderungen so gestaltet, dass dieser dem Nutzer sowohl eine ID als auch einen Nutzernamen zuweist. Darüber hinaus werden dem Server für die Weiterverarbeitung auch die aktuellen Koordinaten der Person übergeben. Im Folgenden wird der in diesem Projekt durch Python realisierte Code, in Abschnitte unterteilt und kurz in seiner Funktion erläutert.\\
\\
Im ersten Schritt nach Programmstart wird der User gebeten sich mit dem Befehl \textit{reg} in der Eingabeaufforderung anzumelden. Daraufhin bekommt dieser seine ID, sowie seinen Nutzernamen und Standort übergeben und im Server hinterlegt. Den entsprechenden Ausschnitt des Codes ist im folgenden zu sehen.

\begin{lstlisting}
#Login des Users
def userin(userinput):
    if userinput =="reg":
        getid()
        receive()
        pass
\end{lstlisting}

Gleiches gilt für die anschließende Bestellung eines der Services. Der Nutzer ist über den Eingabebefehl \textit{order(geforderter Service)} in der Lage den geforderten Service zu rufen. Hierbei stehen folgende Fahrzeugtypen zur Auswahl.

\begin{itemize}

	\item taxi
	\item police
	\item ambulance
	\item firefighter

\end{itemize}

Die damit aufgerufene Methode, in diesem Beispiel für das Taxi, \textit{def ordertaxi()} übergibt die damit zusammenhängenden Daten des Users, mittels des Aufrufs \textit{send(…)} an den Server.
Hier ist die Klasse des Taxis in der Lage sich die notwendigen Informationen auszulesen.

\begin{lstlisting}
#Call For Taxi
def ordertaxi():
    global id
    global cartype
    data1 = {
        "type": "taxi",
        "id": id,
        "coordinates": coordinates
        }
    cartype = "taxi"
    time.sleep(5)
    send(json.dumps(data1),"hshl/mqtt_exercise/user/"+str(id))
\end{lstlisting}

In den weiteren Schritten wird der User direkt mit dem Taxi verbunden um miteinander zu kommunizieren. Dies entlastet zum einen den Server und hat zum anderen den Vorteil die Ausführungszeiten zu verringern. Das Taxi wird sich mitteilen sobald es den Nutzer erreicht hat und dieser wiederum ein weiteres Paar Koordinaten als sein Ziel angeben. Nach erreichen der vom User geforderten Position kann das Servicefahrzeug von eben diesem wieder als \textit{free} gesetzt werden und kann sich um weitere Aufträge anderer Nutzer kümmern.

\begin{lstlisting}
#Seting Vehicle Status To Free
def setToFree():
    global id
    global idCar
    global cartype
    print("SERVICE VEHICLE SET TO FREE")
    data = {
    "type": cartype,
    "id": id,
    "idCar": idCar,
    }
    send(json.dumps(data),"hshl/mqtt_exercise/user/"+str(id)+"/status/reset")
\end{lstlisting}


\section{Fazit}\label{fazit}
Das Ziel dieses Projekts bestand darin ein Programm zu entwickeln, welches dazu dient eine Person (hier auch User oder Nutzer) mithilfe einfachster Schritte einen speziellen Service in Anspruch nehmen zu lassen, welche sich aus dem vorherigen Kapitel unter dem Begriff Servicefahrzeuge entnehme lassen. Hierbei wurden diverse Anforderungen schon vorher festgelegt und definierten den Projektrahmen. So wurde gefordert, dass jeder Nutzer dieser Anwendung, mittels einer eigenen ID identifiziert werden kann und zudem seinen Namen und seine aktuellen Koordinaten an den Server schickt. Der Server wiederum diente der Verarbeitung und Speicherung der erhaltenen Nachrichten, um so eine Art 'Schwarzes Brett' für alle Services darzustellen. Die Servicefahrzeuge waren somit in der Lage sich die nötigen Informationen zu holen und mit dem Nutzer in Kontakt zu treten.\\
Im Bezug auf die Zukunft und die mögliche Anwendbarkeit einer solchen Anwendung lässt sich bisher nur spekulieren. Für einfache Services wie Taxen, welches ebenfalls eine Dienstleistung in dieser Anwendung ist, ist diese Art von Technik bereits durch ein namhaftes Unternehmen weltweit vertreten. Hinsichtlich der zur Verfügung Stellung von Polizei, Krankenwagen und Feuerwehr müssen jedoch noch viele Ressourcen in ein flächendeckendes und einheitliches System gesteckt und unter Berücksichtigung der Sicherheit beurteilt werden.\\
In diesem Fall war das Projekt jedoch ein Erfolg und könnte die Basis für ein eben solches System darlegen. Durch die gute Koordinierung und Rücksprache ließen sich Probleme und Anregungen schnell besprechen und in der Gruppe lösen.


\end{document}
