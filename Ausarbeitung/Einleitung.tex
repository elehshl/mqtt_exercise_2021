\section{Einleitung}


Im laufe der Jahre, findet MQTT vielseitigen Einsatz, für Sensordaten, 
klassische Nachrichten, Aktienkurse oder Kurzmitteilungen bei der Facebook Mobil App.


Dies weckt Interesse bei vielen Nutzern, 
sich mit dem Thema MQTT vertraut zu machen.


MQTT ist ein Client-Server-Protokoll.
 

Clients senden dem Server (Broker) 
nach Verbindungsaufbau Nachrichten mit einem Topic, 

welches die versendeten Nachrichten strukturiert. 

Zum Beispiel Haus/Wohnzimmer/Sofa.


So hat sich unsere Gruppe die Aufgabe gestellt, 
das MQTT Projekt MQTTexercise2021 zu 

programmieren und so Kenntnisse zum Thema Mqtt zu sammeln.


In dem Projekt haben wir uns die Aufgabe gestellt, 
einen Server zu Programmieren der Nachrichten von anderen Clients (User, Services, Taxi) empfängt und Nachrichten versenden kann.


\subsection{Aufgaben}

Server: Registriert alle Clients, verteilt IDS, übermittelt Nachrichten, 
erhält Nachrichten, leitet/organisiert den Ablaufplan.


User: Anmelden beim Server, order Fahrzeug, sendet Koordinaten, Fahrzeug kommt, wird zum Zielort gebracht, setzt Car free


Taxi: Anmeldung Server, fährt zu den Koordinaten, 

übermittelt Koordinaten, übermittelt Nachrichten, erhält Nachrichten.


Services: Anmelden beim Server, sendet Koordinaten, 

warten auf Nachricht, 

bekommt Koordinaten von User, zum User fahren, 

Fahrzeug ist  bussy, User zum Ziel fahren, Fahrzeug wieder free

\subsection{Ziel}

Das Ziel des Projektes ist, ein Programm zu schreiben, welches ein MQTT Ablauf zeigt und eine funktionierende Kommunikation zwischen den Clients enthält.

\subsection{Ablauf}
Zunächst wird in Kapitel 2 die Projektaufgabe beschrieben. In Kapitel 3 werden die Vorbereitungen und Recherche dargestellt und in Kapitel 4 der Aufbau des Projektes. Die Arbeit endet mit einem Fazit. 






