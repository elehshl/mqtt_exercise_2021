\section{Abstract}
Die folgende Dokumentation befasst sich mit den Anforderungen und der Umsetzung der MQTT-Exercise. Die einzelnen Kapitel werden sich mit der Herangehensweise und der Umsetzung dieser Aufgabe befassen.\\
Zu Beginn wird die Einleitung ein paar Worte über die geforderte Übung geben, sowie die gestellten Anforderungen und jene die sich im weiteren Verlauf der Vorbereitung ergeben haben.
Im nächsten Schritt werden die Diagramme, welche genutzt wurden um die Rahmenbedingungen festzulegen, näher erläutert. Die hier gezeigten Grafiken befassen sich unter anderem mit der ausgearbeiteten Anforderungsliste, aber auch den Use-Case Diagrammen, um festzulegen welche Form von Nachricht wie und wohin gesendet wird.\\
Anschließend werden die uns zur Verfügung stehenden, sowie auch geforderten Arbeitswerkzeuge dargelegt. Dieses Kapitel wird einen kleinen Einblick in MQTT liefern, aber auch weitere Software wie GitHub und Atom einbeziehen. Des Weiteren soll hier erläutert werden, warum dieses Projekt mit bestimmter Software gelöst worden ist.\\
Nachdem die Grundlegenden Arbeitsmethoden definiert sind, wird das darauffolgenden Kapitel sich mit der Anmeldung eines Clients und der schlussendlichen Bestellung eines Servicefahrzeugs befassen. Dieses Kapitel wird den Kern dieser Dokumentation bilden und die finale Umsetzung des Projekts beschreiben.\\
Am Ende dieser Arbeit wird mit einem Fazit noch einmal resümierend auf die Aufgabe zurückgeschaut, aber auch ein Ausblick auf die mögliche Zukunft solcher Programme geworfen. Zusätzlich werden noch negative, aber auch positive Aspekte während des Projekts betrachtet.\\
Schlussendlich sein noch erwähnt, dass sich zusätzlich zu den einzelnen Kapiteln, noch Teile dieser Arbeit mit den Codes der verschiedenen Teilnehmer befasst wird. Diese Abschnitte sollen einen Einblick in das Backend dieses Projekts werfen und für Transparenz sorgen, sowie das nötige Verständnis für bestimmte Anforderungen und Entscheidungen liefern.